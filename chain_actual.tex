% load the library when editing in QTikZ, comment out for inclusion in the paper
%\usetikzlibrary{positioning}
\usetikzlibrary{shapes.geometric}

\begin{tikzpicture}[
	nginx/.style={
		shape=rectangle,
		draw,
		rounded corners,
		minimum width=50mm,
		minimum height=10mm,
		fill=white,
	},
	appserver/.style={
		shape=rectangle,
		draw,
		rounded corners,
		minimum width=15mm,
		minimum height=10mm,
		text width=15mm,
		align=center,
	},
	db/.style={
		shape=cylinder,
        shape border rotate=90,
		aspect=0.25,
		draw
	},
	>=stealth,
	->,
	thick,
	font=\footnotesize,
	node distance=7mm,
]
% draw the bounding rectangle
%\draw (0,0) rectangle (8.45, 5);

\draw [dashed, >=, opacity=0.4] (0.5, 1) node[above right, opacity=1] {Local} node[below right, opacity=1] {External} -- (7.95, 1);

\node (nginx) [nginx]  at (4.75, 1) {NGINX};
\node (django) [appserver, above=of nginx, xshift=-10mm, yshift=10mm] {Django Daemon};
\node (flask) [appserver, above=of nginx, xshift=20mm, yshift=10mm] {Flask Daemon};
\node [db, left of=django, xshift=-15mm] (db) {Postgres};

\draw (nginx) -- node[left, text width=14mm] {HTTP Requests} (django);
\draw (nginx) -- node[right] {Websockets} (flask);
\draw (django) -- node[above] {ZMQ} (flask);
\draw (django) -- node[above] {} (db);



\end{tikzpicture}
