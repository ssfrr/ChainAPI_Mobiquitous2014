% THIS IS SIGPROC-SP.TEX - VERSION 3.1
% WORKS WITH V3.2SP OF ACM_PROC_ARTICLE-SP.CLS
% APRIL 2009
%
% It is an example file showing how to use the 'acm_proc_article-sp.cls' V3.2SP
% LaTeX2e document class file for Conference Proceedings submissions.
% ----------------------------------------------------------------------------------------------------------------
% This .tex file (and associated .cls V3.2SP) *DOES NOT* produce:
%       1) The Permission Statement
%       2) The Conference (location) Info information
%       3) The Copyright Line with ACM data
%       4) Page numbering
% ---------------------------------------------------------------------------------------------------------------
% It is an example which *does* use the .bib file (from which the .bbl file
% is produced).
% REMEMBER HOWEVER: After having produced the .bbl file,
% and prior to final submission,
% you need to 'insert'  your .bbl file into your source .tex file so as to provide
% ONE 'self-contained' source file.
%
% Questions regarding SIGS should be sent to
% Adrienne Griscti ---> griscti@acm.org
%
% Questions/suggestions regarding the guidelines, .tex and .cls files, etc. to
% Gerald Murray ---> murray@hq.acm.org
%
% For tracking purposes - this is V3.1SP - APRIL 2009

\documentclass{acm_proc_article-sp}

% command from http://www.acm.org/sigs/publications/sigfaq
\def\sharedaffiliation{
\end{tabular}
\begin{tabular}{c}}

\begin{document}

\title{Hypermedia APIs for Sensor Data}
\subtitle{A pragmatic approach to the Web of Things}

% You need the command \numberofauthors to handle the 'placement
% and alignment' of the authors beneath the title.
%
% For aesthetic reasons, we recommend 'three authors at a time'
% i.e. three 'name/affiliation blocks' be placed beneath the title.
%
% NOTE: You are NOT restricted in how many 'rows' of
% "name/affiliations" may appear. We just ask that you restrict
% the number of 'columns' to three.
%
% Because of the available 'opening page real-estate'
% we ask you to refrain from putting more than six authors
% (two rows with three columns) beneath the article title.
% More than six makes the first-page appear very cluttered indeed.
%
% Use the \alignauthor commands to handle the names
% and affiliations for an 'aesthetic maximum' of six authors.
% Add names, affiliations, addresses for
% the seventh etc. author(s) as the argument for the
% \additionalauthors command.
% These 'additional authors' will be output/set for you
% without further effort on your part as the last section in
% the body of your article BEFORE References or any Appendices.

\numberofauthors{2}

\author{
% You can go ahead and credit any number of authors here,
% e.g. one 'row of three' or two rows (consisting of one row of three
% and a second row of one, two or three).
%
% The command \alignauthor (no curly braces needed) should
% precede each author name, affiliation/snail-mail address and
% e-mail address. Additionally, tag each line of
% affiliation/address with \affaddr, and tag the
% e-mail address with \email.
%
% 1st. author
\alignauthor Spencer Russell\\
    \email{sfr@media.mit.edu}
% 2nd. author
\alignauthor Joseph A. Paradiso\\
    \email{joep@media.mit.edu}
\sharedaffiliation
    \\
    \affaddr{Responsive Environments Group}  \\
    \affaddr{MIT Media Lab}   \\
    \affaddr{Massachusetts Institute of Technology} \\
    \affaddr{Cambridge, MA, USA}
}

\date{14 July 2014}

\maketitle

\begin{abstract}
In this paper we introduce a pragmatic approach to the Web of Things that is
inspired equally by the rich body of Semantic Web research and industry web
services practices. This approach integrates HTTP request/response interactions
with realtime streaming using HTML5 WebSockets. We will also describe our
implementation of these concepts in ChainAPI, a sensor data server in use by
a variety of client applications.
\end{abstract}

% % A category with the (minimum) three required fields
% \category{H.4}{Information Systems Applications}{Miscellaneous}
% %A category including the fourth, optional field follows...
% \category{D.2.8}{Software Engineering}{Metrics}[complexity measures, performance measures]
%
% \terms{Theory}
%
\keywords{Semantic Web, RESTful Web Services, Sensors, Internet of Things, Hypermedia} % NOT required for Proceedings

\section{Introduction}

It is becoming apparent that in addition to a transport layer that enables the
so-called Internet of Things, it is important to develop an application layer
to provide wide-spread interoperability and a consistent interface to
internet-connected devices. While there are many efforts [cite Alljoyn, etc.]
to develop new standards and protocols, other projects [cite] seek to use
existing application-level web standards such as HTTP to provide an interface
that is more familiar to developers, and also that can take advantage of
tooling and infrastructure already in place for the World Wide Web. Reflecting
the relationships to existing web standards and also the way in which the World
Wide Web is built on top of the Internet, these efforts are often dubbed the
Web of Things.

In previous work~\cite{doppellab}\cite{gestures} we have built frameworks to collect
and process sensor data from a variety of sources, as well as applications to
visualize and experience those data. [Expand on how this informed ChainAPI]

We posit that the main impediments to adoption of IoT standards are social
rather then technological. Often  solutions require developers to take on too much
simultaneous complexity to get started. Many of the systems coming from a Semantic
Web history have sophisticated data models to ensure compatibility with existing
upper ontologies [cite]. Application developers are often unwillinging to adopt
this additional complexity [why?].

To address these issues we have developed and implemented an API that interoperates
easily with existing infrastructure, and also allows developers to take advantage
of semantic relations and formal ontologies without requiring it.

\section{Related Work}

I'm sure some people have written about this stuff.

\section{Bringing Poll and Push\\ Together}

Many existing systems [citation needed] focus on either a RESTful HTTP API in
which clients make a request and the server sends a response, or are pub/sub
systems supporting real-time streaming data to be pushed to clients as it is
available.

\section{Enabling Hierarchy}

Work is under way by multiple groups to adapt the TCP/IP Stack to be more
suitable for low-power and resource-constrained devices~\cite{iotsurvey}.
Though this is a reasonable proposition and would provide a suitable transport
protocol for communication, it leaves open many questions that are important
for secure and reliable communication over the open internet. Even if the
devices themselves speak IP (whether using WiFi, 6LoWPAN, etc.) there will
still likely be a role for a bridge or gateway node that can handle encryption,
authentication, discovery mechanisms, and other application logic necessary to
communicate and interoperate with the larger Internet.

One of the benefits of this approach is that we can take advantage of existing
HTTP Caching and Proxy infrastructure. It is a common pattern in modern web
development to have application web server processes handling application
logic, and to place a front-end HTTP server such as Nginx or Apache as a proxy.
In this configuration, the proxy is responsible for handling SSL, defending
against DDOS attacks, and in general provides a front line of defense to the
open internet. The application server processes (e.g. Node.js or Gunicorn) thus
operate in a safer environment and focus on handling application logic.

expand on how we can use this same model, and perhaps even use some of the
same infrastructure

\section{Naming "Things"}

One of the central issues in the IoT is simply the issue of identifying objects
in the system~\cite{iotsurvey}. We propose the use of HTTP URIs as globally
unique identifiers.  Providers can structure their URIs aritrarily, for
instance to represent natural hierarchy in the system. Link relations between
objects not only represent identity, but also where the linked object can be
found, without needing to first consult any sort of central registry.

Additionally, HTTP has built-in mechanisms to handle renaming, as servers can
respond with an HTTP Status 301 (Moved Permanently) to notify clients that the
object can now be found at a new URI.

\section{Using Existing Standards}

Wherever possible we have relied on existing standards and protocols rather
than reinventing our own. For instance, to support hypermedia in our responses
we are using the Hypertext Application Language~\cite{json-hal-draft}, which
provides a standardized data model for hyperlinks that can be rendered in JSON
or XML (though JSON is the main focus and was implemented by the authors).

\section{Supporting a Relation Ontology}

ChainAPI is inspired by the rich literature and history of the Semantic Web.
This section needs to talk about how our links can support various ontologies,
and talk a little about ours.

\section{Usage Examples}

Our ChainAPI implementation is currently in use as the back-end infrastructure
for several ongoing projects in research and industry. In general these applications
have the following structure: [describe system roles, maybe a diagram...]

\subsection{Tidmarsh Living Observatory}

Talk about what the system does, how it handles data, compare against previous
implementation of doppellab.

\subsection{Soofa Solar Benches}

Maybe focus on how quickly they got up and running?

\section{Conclusion}
And here we conclude, with some stirring remarks.

%ACKNOWLEDGMENTS are optional
\section{Acknowledgments}
This project receives partial funding from Cisco Systems, whom we would like to
thank.

% The following two commands are all you need in the
% initial runs of your .tex file to
% produce the bibliography for the citations in your paper.
\bibliographystyle{abbrv}
\bibliography{refs}  % sigproc.bib is the name of the Bibliography in this case
%
% ACM needs 'a single self-contained file'! For final submission paste the contents
% of refs.bbl here.
%
\end{document}
